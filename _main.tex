\documentclass[]{book}
\usepackage{lmodern}
\usepackage{amssymb,amsmath}
\usepackage{ifxetex,ifluatex}
\usepackage{fixltx2e} % provides \textsubscript
\ifnum 0\ifxetex 1\fi\ifluatex 1\fi=0 % if pdftex
  \usepackage[T1]{fontenc}
  \usepackage[utf8]{inputenc}
\else % if luatex or xelatex
  \ifxetex
    \usepackage{mathspec}
  \else
    \usepackage{fontspec}
  \fi
  \defaultfontfeatures{Ligatures=TeX,Scale=MatchLowercase}
\fi
% use upquote if available, for straight quotes in verbatim environments
\IfFileExists{upquote.sty}{\usepackage{upquote}}{}
% use microtype if available
\IfFileExists{microtype.sty}{%
\usepackage{microtype}
\UseMicrotypeSet[protrusion]{basicmath} % disable protrusion for tt fonts
}{}
\usepackage{hyperref}
\hypersetup{unicode=true,
            pdftitle={Agile Data Science with R},
            pdfauthor={Edwin Thoen},
            pdfborder={0 0 0},
            breaklinks=true}
\urlstyle{same}  % don't use monospace font for urls
\usepackage{natbib}
\bibliographystyle{apalike}
\usepackage{longtable,booktabs}
\usepackage{graphicx,grffile}
\makeatletter
\def\maxwidth{\ifdim\Gin@nat@width>\linewidth\linewidth\else\Gin@nat@width\fi}
\def\maxheight{\ifdim\Gin@nat@height>\textheight\textheight\else\Gin@nat@height\fi}
\makeatother
% Scale images if necessary, so that they will not overflow the page
% margins by default, and it is still possible to overwrite the defaults
% using explicit options in \includegraphics[width, height, ...]{}
\setkeys{Gin}{width=\maxwidth,height=\maxheight,keepaspectratio}
\IfFileExists{parskip.sty}{%
\usepackage{parskip}
}{% else
\setlength{\parindent}{0pt}
\setlength{\parskip}{6pt plus 2pt minus 1pt}
}
\setlength{\emergencystretch}{3em}  % prevent overfull lines
\providecommand{\tightlist}{%
  \setlength{\itemsep}{0pt}\setlength{\parskip}{0pt}}
\setcounter{secnumdepth}{5}
% Redefines (sub)paragraphs to behave more like sections
\ifx\paragraph\undefined\else
\let\oldparagraph\paragraph
\renewcommand{\paragraph}[1]{\oldparagraph{#1}\mbox{}}
\fi
\ifx\subparagraph\undefined\else
\let\oldsubparagraph\subparagraph
\renewcommand{\subparagraph}[1]{\oldsubparagraph{#1}\mbox{}}
\fi

%%% Use protect on footnotes to avoid problems with footnotes in titles
\let\rmarkdownfootnote\footnote%
\def\footnote{\protect\rmarkdownfootnote}

%%% Change title format to be more compact
\usepackage{titling}

% Create subtitle command for use in maketitle
\providecommand{\subtitle}[1]{
  \posttitle{
    \begin{center}\large#1\end{center}
    }
}

\setlength{\droptitle}{-2em}

  \title{Agile Data Science with R}
    \pretitle{\vspace{\droptitle}\centering\huge}
  \posttitle{\par}
  \subtitle{A workflow}
  \author{Edwin Thoen}
    \preauthor{\centering\large\emph}
  \postauthor{\par}
    \date{}
    \predate{}\postdate{}
  
\usepackage{booktabs}

\begin{document}
\maketitle

{
\setcounter{tocdepth}{1}
\tableofcontents
}
\hypertarget{working-without-a-workflow}{%
\chapter{Working without a Workflow}\label{working-without-a-workflow}}

When I was starting my career as a data scientist, I did not really have a workflow.
Freshly out of statistics grad school I entered the arena of Dutch business, employed by a small consulting firm.
Between the company, the potential clients and myself, no one knew what it meant to implement a statistical model or a machine learning method in the ``real'' world.
But everybody was interested in this ``Big Data'' thing, so we quickly started to do consulting work without a clear idea what I was going to do.
When we came to something that looked like a project, I plunged into it.
Eager to deliver results quickly, I loaded data extracts into R and started to apply all kinds of different models and algorithms on it.
Findings ended up in the code comments of the R scripts, scripts that often grew to hundreds or even thousands of lines.

The only system I had was numbering the scripts sequentially.
Soon I found myself amidst dozens of scripts and data exports of intermediate results that were no longer reproducible.
The R session I was running \emph{ad infinitum} was sometimes mistakenly closed, or it crashed (which was bound to happen as the memory used grew).
When this happened, I spent hours or even days to recreate the results.
Deadlines were a nightmare; everything I had done up to that point had to be loaded, joined and compared at the last moment.
More often than not, the model results were different from that noted earlier, with no indication if I was mistaken earlier, I was using the wrong data now, or some other source of error was introduced.
Looking back, I had no clue about the importance of a good workflow for doing larger data science projects.
Several times I was saved when the plug was pulled from the project for other reasons, saving me from the embarrassment of not being able to deliver.

I have learned a great deal since those days, both from the insights shared by others and from my own experiences.
Writing an R package to be shipped to CRAN enforced me to understand the basics of software engineering.
Not being able to reproduce crucial results forced me to start thinking about end-to-end research and model building, controlling all the steps along the way.
Last year, for the first time, I joined a Scrum team (frontend, backend, ux designer, product owner, second data scientist) to create a machine learning model that we brought to production using the Agile principles.
It was an inspiring experience from which I learned a great deal.
My colleagues patiently explained the principles of Agile software development and together we applied them to the data science context.

\hypertarget{what-this-text-is-about}{%
\section{What this Text is About}\label{what-this-text-is-about}}

All these experiences culminated in the workflow that we now adhere to at work and I think it is worthwhile to share it.
It is heavily based on the principles of Agile software production, hence the title.
We have explored which of the concepts from Agile did and did not work for data science and we got hands-on experience in working from these principles in an R project that actually got to production.
The story of how we created \emph{Valuecheck} can be found as an Appendix in the final chapter.
This text is split into two parts.
In the first we will look into the Agile philosophy and some of the methodologies that are closely related to it (chapters 2 and 3).
Both will be related to the data science context, seeing what we can get from the philosophy (chapter 4) and what an Agile machine learning workflow might look like (chapter 5).
The second part is hands on. We will explore how we can leverage the possibilities in the R software system to implement Agile data science.

\hypertarget{limitations}{%
\section{Limitations}\label{limitations}}

Data science projects can greatly differ from each other.
There are so many variables that make projects unique; its goal, its type (deriving insights, machine learning, building as dashboard), the data quality, and the expertise of the data scientist(s).
This implies that there are necessarily aspects of data science projects experienced by others that I am not aware of.
In this text I am relating my own experiences to the theory and best practises of Agile software development to come up with a general workflow.
This means that I am probably ``overfitting'' the workflow on the dozen or so large data science projects I have done.
If you feel that what I write is not broadly applicable, or if you think there are topics overlooked, please file an \href{https://github.com/EdwinTh/ADSwR}{issue}.

This text is meant to be a living thing with the objective of documenting a workflow that yields optimal reproducibility, quick shipping of results and high quality code.
The more people share their best practises, the closer we get to this objective.
Please follow along on this journey and get involved!
Finally, I am not a native English speaker so fixed typos and style improvements are greatly appreciated.

\hypertarget{intended-audience}{%
\section{Intended Audience}\label{intended-audience}}

The title of this text has four components: \emph{Agile}, \emph{Data Science}, \emph{R}, and \emph{Workflow}.
If you are interested in all four, you're obviously in the right place.
This text is not for you if you hope to learn about different algorithms and statistical techniques to do data science; more knowledgeable people have written many books and articles on those topics.
Also it will not teach you anything about R programming.
The workflow I present is completely separate from the algorithms you choose and the data preparation tools of your preference, as it focuses on code organisation and delivery.
If you use python rather than R, you will still find this text valuable, especially the first part, which focuses on workflow only and is tool agnostic.

The larger data science projects I was involved with all had the objective of delivering predictions in some way, so you can file them under machine learning.
I intend to present a generic workflow that is also applicable to data science projects that have a different type of delivery, such as automated reports and Shiny applications.
You might find machine learning a bit overrepresented in the examples and applications.
If you think there is still a misfit between your daily data science practice, please let me know.

\hypertarget{contributors}{%
\section{Contributors}\label{contributors}}

The following people contributed to this text, by suggesting improvements or doing pull requests. Thank you!

Arttu Kosonen(\texttt{@datarttu}), Colin Fay(\texttt{@ColinFay}), Dilsher Singh Dhillon(\texttt{@dshelldhillon}), Jesse Tweedle(\texttt{@tweed1e}), Lorenz Walthert(\texttt{@lorenzwalthert}), Nathan Moore(\texttt{@nmoorenz}), Nic Fox(\texttt{@foxnic})

\hypertarget{part-agile-data-science}{%
\part{Agile Data Science\ldots{}}\label{part-agile-data-science}}

\hypertarget{agile-in-a-nutshell}{%
\chapter{Agile in a Nutshell}\label{agile-in-a-nutshell}}

Placeholder

\hypertarget{the-origins-of-agile}{%
\section{The Origins of Agile}\label{the-origins-of-agile}}

\hypertarget{the-manifesto}{%
\section{The Manifesto}\label{the-manifesto}}

\hypertarget{the-twelve-principles}{%
\section{The Twelve Principles}\label{the-twelve-principles}}

\hypertarget{agile-methods}{%
\chapter{Agile Methods}\label{agile-methods}}

Placeholder

\hypertarget{scrum}{%
\section{Scrum}\label{scrum}}

\hypertarget{scrum-roles}{%
\subsection{Scrum Roles}\label{scrum-roles}}

\hypertarget{ceremonies}{%
\subsection{Ceremonies}\label{ceremonies}}

\hypertarget{kanban}{%
\section{Kanban}\label{kanban}}

\hypertarget{scrum-vs-kanban}{%
\section{Scrum vs Kanban}\label{scrum-vs-kanban}}

\hypertarget{agile-data-science}{%
\chapter{Agile Data Science}\label{agile-data-science}}

Placeholder

\hypertarget{data-science-waterfall}{%
\section{\texorpdfstring{Data Science \emph{Waterfall}}{Data Science Waterfall}}\label{data-science-waterfall}}

\hypertarget{the-twelve-principle-in-the-data-science-context}{%
\section{The Twelve Principle in the Data Science Context}\label{the-twelve-principle-in-the-data-science-context}}

\hypertarget{in-summary}{%
\section{In Summary}\label{in-summary}}

\hypertarget{a-methodology-for-agile-data-science}{%
\chapter{A Methodology for Agile Data Science}\label{a-methodology-for-agile-data-science}}

Placeholder

\hypertarget{linear-and-circular-tasks}{%
\section{Linear and Circular Tasks}\label{linear-and-circular-tasks}}

\hypertarget{a-two-way-workflow-for-development}{%
\section{A Two-Way Workflow for Development}\label{a-two-way-workflow-for-development}}

\hypertarget{formulating-tasks}{%
\section{Formulating Tasks}\label{formulating-tasks}}

\hypertarget{using-kanban-for-data-science}{%
\section{Using Kanban for Data Science}\label{using-kanban-for-data-science}}

\hypertarget{scoping-tasks}{%
\section{Scoping Tasks}\label{scoping-tasks}}

\hypertarget{the-product-owner-role}{%
\section{The Product Owner Role}\label{the-product-owner-role}}

\hypertarget{part-with-r}{%
\part{\ldots{}with R}\label{part-with-r}}

\hypertarget{code-organsation}{%
\chapter{Code Organsation}\label{code-organsation}}

Placeholder

\hypertarget{using-the-r-package-structure}{%
\section{Using the R Package Structure}\label{using-the-r-package-structure}}

\hypertarget{further-reading}{%
\section{Further Reading}\label{further-reading}}

\hypertarget{automation}{%
\chapter{Automation}\label{automation}}

Placeholder

\hypertarget{efficiency}{%
\subsection{Efficiency}\label{efficiency}}

\hypertarget{automation-is-documentation}{%
\subsection{Automation is Documentation}\label{automation-is-documentation}}

\hypertarget{reproducibility}{%
\subsection{Reproducibility}\label{reproducibility}}

\hypertarget{how-to-automate}{%
\section{How to Automate}\label{how-to-automate}}

\hypertarget{code-that-fails}{%
\chapter{Code that Fails}\label{code-that-fails}}

Placeholder

\hypertarget{assumptions-to-your-data}{%
\section{Assumptions to your Data}\label{assumptions-to-your-data}}

\hypertarget{type-checking}{%
\subsection{Type Checking}\label{type-checking}}

\hypertarget{column-validation}{%
\subsection{Column Validation}\label{column-validation}}

\hypertarget{data-validation}{%
\subsection{Data Validation}\label{data-validation}}

\hypertarget{unit-testing}{%
\section{Unit Testing}\label{unit-testing}}

\hypertarget{externalising-assumptions}{%
\subsection{Externalising Assumptions}\label{externalising-assumptions}}

\hypertarget{writing-better-code}{%
\subsection{Writing Better Code}\label{writing-better-code}}

\hypertarget{versioning}{%
\chapter{Versioning}\label{versioning}}

Placeholder

\hypertarget{old-code}{%
\section{Old Code}\label{old-code}}

\hypertarget{version-control}{%
\section{Version Control}\label{version-control}}

\hypertarget{master-is-production}{%
\subsection{Master is Production}\label{master-is-production}}

\hypertarget{working-together}{%
\subsection{Working Together}\label{working-together}}

\hypertarget{exploratory-tasks}{%
\chapter{Exploratory Tasks}\label{exploratory-tasks}}

We have explored the aspects of creating a data product in Agile data science workflow. The second part, the exploratory part, does not need so much attention, because by definition you can do what you like in her. One aspect deserves some consideration.

\hypertarget{unreproducible-research}{%
\section{(Un)Reproducible Research}\label{unreproducible-research}}

In the chapter on versioning it was advocated that you should not keep old code lingering as you move on.
Rather use version control systems to keep track of the changes you are making to the product.
Research tasks are explored in vignettes, these are kept around as the project goes along.
It is likely that in research reports code from the software product and data from the pipeline is used.
These are readily available, enabling quick exploration of hypotheses.
However, as you move to next versions, the software product and the data will be updated.
This means that the reproducibility of almost all research reports will be lost at some point.

It is up to you and the characteristics of the project if this is considered a problem.
If you want to keep all reports to be reproducible at any time, there is a considerable amount of extra work to the research tasks.
Data has to be stored separately and the source code used should be frozen somewhere outside the \texttt{\textbackslash{}R} folder.
I prefer to accept that the reports will get out of sync with the data and the source code.
The conclusions are what matter most, and the code is always around to check if they were derived properly, even if it can no longer be run.
If you really have to rerun the research report to verify results, you can always checkout the commit of the report and rerun the pipeline.
The source code and state of the data at that ``time point'' are then available.
I tend to number my research scripts to the version of the data product of that moment.
This will make it easy to understand the context in which the research was done at a later time point.

\hypertarget{part-miscellaneous}{%
\part{Miscellaneous}\label{part-miscellaneous}}

\hypertarget{tldr}{%
\chapter{TL;DR}\label{tldr}}

Placeholder

\hypertarget{case-study-building-the-valuecheck}{%
\chapter{\texorpdfstring{Case Study: Building the \emph{Valuecheck}}{Case Study: Building the Valuecheck}}\label{case-study-building-the-valuecheck}}

Placeholder

\hypertarget{trying-scrum}{%
\section{Trying Scrum}\label{trying-scrum}}

\hypertarget{informing-the-business}{%
\section{Informing the Business}\label{informing-the-business}}

\hypertarget{moving-to-kanban}{%
\section{Moving to Kanban}\label{moving-to-kanban}}

\hypertarget{building-an-mvm-for-the-mvp}{%
\section{Building an MVM for the MVP}\label{building-an-mvm-for-the-mvp}}

\hypertarget{improving-the-product}{%
\section{Improving the Product}\label{improving-the-product}}

\hypertarget{interactivity}{%
\subsection{Interactivity}\label{interactivity}}

\hypertarget{changing-the-model}{%
\subsection{Changing the Model}\label{changing-the-model}}

\hypertarget{productionising-the-interactive-model}{%
\section{Productionising the Interactive Model}\label{productionising-the-interactive-model}}

\hypertarget{thank-you}{%
\section{Thank You!}\label{thank-you}}

\bibliography{book.bib,packages.bib}


\end{document}
